%!TEX root = ../csuthesis_main.tex
% 论文正文是主体,主体部分应从另页右页开始,每一章应另起页。一般由序号标题、文字叙述、图、表格和公式等五个部分构成。
\section{绪论}
\subsection{研究背景与意义}

目的是创建一个符合中南大学研究生学位论文(博士)撰写规范的LaTeX模板,解决学位论文撰写时格式调整的痛点。

已有珠玉在前,我们之所以还要重新造轮子,主要是参考了2020年4月2号学校下发的[《中南大学研究生学位论文撰写规范》中大研字【2020】30号]( http://oa.its.csu.edu.cn/Home/ReleaseMainText/909CD53BF50943CD97B83C352032FEA4 ),重新修改了页面布局、字体类型和大小、标题内容,以期做到与 Word 模板尽可能的相似。主要修改如下:
\begin{itemize}
\item 增加符号说明页和附录页(如果不需要,请在.cls文件对应处注释掉即可);
\item  增加参考文献按国标 gbt7714-2015要求,只核对了常用的图书、中英文期刊,会议格式,其余未常使用的未进行核对(如有问题请改回gbt7714-2005);
\item  修订多个子图Caption居中问题;
\item 依据专家意见调整成果与致谢部分间距,并增加目录中的点密度;
\item 按照图书馆最新要求(2020年12月份),去除目录中红色边框;
\item 增加页眉信息:中南大学博士论文与右侧的章节名保持一致,以及无需号章节名保持一致;
\item 增加中英文摘要至目录,并保持与章节名昨对其;
\item 参考文献完全依照国标 gbt7714-2005,修正了部分 Bug,提供了新的引用命令;
\item 按照最新版本要求,在声明扉页前后各增加一页空白页,保证装订单独成页;
\item 章节标题居中,并改成‘第1章’样式;
\item 目录中,将原章节标题换成‘第几章’样式,字体按要求加粗;
\item 中文摘要到目录结束用罗马数字编写页码,小五号Times New Roman,居中;
\item 增加插图索引和表格索引;
\item 所有的章节题目和中英文摘要均按要求修改字体和间距;
\end{itemize}

\subsection{主要研究工作}
我堂堂双一流高校竟然没有官方研究生论文LaTeX模板!!!虽然我LaTeX水平也很水……但是通过大量debug也勉强给大家凑出来一个格式绝对标准的LaTeX模板,模板代码丑就丑吧,能用就行。写了大量注释,有一点LaTeX基础就可以根据自己需要修改CSUthesis.cls文件。

(1) 提供图片插入示例。

(2) 提供表格插入示例。

(3) 提供公式插入示例。

(4) 提供参考文献插入示例。

\subsection{论文组织结构}

全文内容共六章,具体内容组织如下:

第一章为绪论。

第二章为图片插入示例。

第三章为表格插入示例。

第四章为公式插入示例。

第五章为参考文献插入示例。

第六章总结与展望,总结了本文的主要工作,展望了下一阶段的研究方向。

\newpage

\section{图像布局}


\subsection{单图布局}



\textbf{单图布局如图\ref{F.csu_single}所示。}

\begin{figure}[hbt]
\centering
\includegraphics[width=0.5\textwidth]{csu.png}
\caption{单图布局示例}
\label{F.csu_single}
\end{figure}

\subsection{横排布局}

\textbf{横排布局如图\ref{F.csu_row}所示。}

\begin{figure}[!htb]
    \centering
    \begin{subfigure}[t]{0.24\linewidth}
    	\captionsetup{justification=centering} %子图caption居中
        \begin{minipage}[b]{1\linewidth}
        \includegraphics[width=1\linewidth]{csu.png}
         \caption{}
        \end{minipage}
    \end{subfigure}
    \begin{subfigure}[t]{0.24\linewidth}
    \captionsetup{justification=centering} %子图caption居中
        \begin{minipage}[b]{1\linewidth}
        \includegraphics[width=1\linewidth]{csu.png}
        \caption{}
        \end{minipage}
    \end{subfigure}
    \begin{subfigure}[t]{0.24\linewidth}
    	\captionsetup{justification=centering} %子图caption居中
        \begin{minipage}[b]{1\linewidth}
        \includegraphics[width=1\linewidth]{csu.png}
        \caption{}
        \end{minipage}
    \end{subfigure}
    \begin{subfigure}[t]{0.24\linewidth}
    	\captionsetup{justification=centering} %子图caption居中
        \begin{minipage}[b]{1\linewidth}
        \includegraphics[width=1\linewidth]{csu.png}
        \caption{}
        \end{minipage}
    \end{subfigure}
    \caption{横排布局示例}
    \label{F.csu_row}
\end{figure}



\subsection{竖排布局}
\textbf{竖排布局如图\ref{F.csu_col}所示。}

\begin{figure}[!htb]
    \centering
    \begin{subfigure}[t]{0.15\linewidth}
        \begin{minipage}[b]{1\linewidth}
        \includegraphics[width=1\linewidth]{csu.png}
        \caption{}
        \end{minipage}
    \end{subfigure}\\
    \begin{subfigure}[t]{0.15\linewidth}
        \begin{minipage}[b]{1\linewidth}
        \includegraphics[width=1\linewidth]{csu.png}
        \caption{}
        \end{minipage}
    \end{subfigure}
    \caption{竖排布局示例}
    \label{F.csu_col}
\end{figure}



\subsubsection{竖排多图横排布局}

\begin{figure}[!htb]
    \centering
    \begin{subfigure}[t]{0.13\linewidth}
    	\captionsetup{justification=centering} %子图caption居中
        \begin{minipage}[b]{1\linewidth}
        \includegraphics[width=1\linewidth]{csu.png} \vspace{-1ex} \vfill
        \includegraphics[width=1\linewidth]{csu.png}
         \caption{}
        \end{minipage}
    \end{subfigure}
    \begin{subfigure}[t]{0.13\linewidth}
    	\captionsetup{justification=centering} %子图caption居中
        \begin{minipage}[b]{1\linewidth}
        \includegraphics[width=1\linewidth]{csu.png} \vspace{-1ex} \vfill
        \includegraphics[width=1\linewidth]{csu.png}
        \caption{}
        \end{minipage}
    \end{subfigure}
    \caption{竖排多图横排布局}
    \label{F.csu_col_row}
\end{figure}

\textbf{竖排多图横排布局如图\ref{F.csu_col_row}所示。注意看(a)、(b)编号与图关系。}


\subsubsection{横排多图竖排布局}



\begin{figure}[!htb]
    \centering
    \begin{subfigure}[t]{0.3\linewidth}
    	\captionsetup{justification=centering} %子图caption居中
        \begin{minipage}[b]{1\linewidth}
        \includegraphics[width=0.45\linewidth]{csu.png}
        \includegraphics[width=0.45\linewidth]{csu.png}
        \caption{}
        \end{minipage}
    \end{subfigure}\\
    \begin{subfigure}[t]{0.3\linewidth}
    	\captionsetup{justification=centering} %子图caption居中
        \begin{minipage}[b]{1\linewidth}
        \includegraphics[width=0.45\linewidth]{csu.png}
        \includegraphics[width=0.45\linewidth]{csu.png}
        \caption{}
        \end{minipage}
    \end{subfigure}
    \caption{横排多图竖排布局,斜体emph \emph{A},A,斜体texit \textit{A}}
    \label{F.csu_row_col}
\end{figure}

\textbf{横排多图竖排布局如图\ref{F.csu_row_col}所示。注意看(a)、(b)编号与图关系。}

\subsection{本章小结}
本章示例图片布局。

\newpage


\section{表格插入示例}

\begin{table}[htb]
  \centering
  \caption{表格为三线表斜体emph \emph{A},A,斜体texit \textit{A}}
  \label{T.example}
  \begin{tabular}{llllll}
  	\toprule
   & \emph{A}A \textit{A}  & B  & C  & D  & E \\
  \toprule
1 	& 212 & 414 & 4 		& 23 & fgw	\\
2 	& 212 & 414 & v 		& 23 & fgw	\\
3 	& 212 & 414 & vfwe		& 23 & 嗯	\\
4 	& 212 & 414 & 4fwe		& 23 & 嗯	\\
5 	& af2 & 4vx & 4 		& 23 & fgw	\\
6 	& af2 & 4vx & 4 		& 23 & fgw	\\
7 	& 212 & 414 & 4 		& 23 & fgw	\\
\bottomrule

\end{tabular}
\end{table}




\textbf{表格如表\ref{T.example}所示,latex表格技巧很多,这里不再详细介绍。}


\newpage

\section{算法示例}


\begin{algorithm}  
	\caption{Fourier-Mellin Based KCF}  
	\label{alg:A}  
	\hspace*{0.02in}{\bf Input:}
	Image $I$\\preprocessed kernelized template $T_\kappa$\\
	\hspace*{0.02in}{\bf Output:} 
	scale $\sigma$, angle $\theta$ relation between $I$ and $T$ 
	
	\begin{algorithmic}[1] 
		\State {fourier transform: $F=\mathcal{F}(I)$}
		\State {high pass filter: $F_h=\mathcal{H}(F)$\\$\mathcal{H}(x,y)=(1.0-cos(\pi x)cos(\pi y))(2.0-cos(\pi x)cos(\pi y))$}
		\State {log-polar transform: $F_{lp}=\mathcal{L}(F_h)$}
		\State {apply kernel function: $F_\kappa=\mathcal{K}(F_{lp})$}
		\State {phase correlation: $(\Delta x, \Delta y)=\mathcal{C}(F_\kappa, T_\kappa)$}
		\State {resolove scale and rotation:\\
			$\theta=\alpha \Delta x$, $\sigma=log(\Delta y)$\\
			where $\alpha$ is translation factor of pixel translation on fourier domain and polar angle on origin image
		}
	\end{algorithmic}  
\end{algorithm}

\newpage

\section{公式插入示例}

\begin{flalign}
\text{P1: } &\min_{\eta,R_u>0,R_d>0}\big\{ T_{\text{latency}}(\eta,R_u,R_d)\big\}  \\
&\text{s.t. }   0 \leq \eta \leq 1 \label{P1C1} 
\end{flalign}

\textbf{公式插入示例如公式(\ref{E.example})所示。}

\begin{equation}
\gamma_{x}=
\left\{
  \begin{array}{lr}
  0, & {\rm if}~~\;|x| \leq \delta \\
  x, & {\rm otherwise}
  \end{array}
\right.
\label{E.example}
\end{equation}

\begin{flalign}
&\text{P1:} \max_{\bigg\{\substack{P_{m,i},P_{n,i}\\
		q_{m,i},q_{n,i}\\ \forall n,m,i}\bigg\}}\big[R_{\text{sum}}(P_{m,i},P_{n,i},q_{m,i},q_{n,i},\forall n,m,i)\big],  \\
& \text{s.t.}~~q_{m,i}\in(0,1), q_{n,i}\in(0,1), \forall n,m,i, \label{allocons1} \\
& \hspace{1.5em} ~0\leq \sum_{i=1}^Rq_{n,i}P_{n,i}\leq P_n^{sum}, \forall n, \label{powercons1}  \\ 
& \hspace{1.5em} ~0\leq \sum_{i=1}^Rq_{m,i}P_{m,i}\leq P_m^{sum}, \forall m, \label{powercons2} \\ 
& \hspace{1.5em} ~\sum_{i=1}^Rq_{n,i}\leq 1,\sum_{i=1}^Rq_{m,i}\leq 1,\forall m,n, \label{RBcons} \\ 
& \hspace{1.5em} ~\sum_{n=1}^Nq_{n,i}\leq 1,\sum_{m=1}^Mq_{m,i}\leq 1,\forall i, \label{RBcons2} \\
& \hspace{1.5em} ~C_{m,BS,i}(P_{m,i},q_{m,i})\geq\varepsilon_{m,i},\forall m, \label{capcons}
\end{flalign}

\newpage

\section{参考文献插入示例}

LaTeX\cite{lamport1994latex}插入参考文献最方便的方式是使用bibliography\cite{pritchard1969statistical},大多数出版商的论文页面都会有导出bib格式参考文献的链接,建议使用Jabref管理参考文献,把每个文献的bib放入``thesis-references'',然后用bibkey即可插入参考文献。

中文文献\cite{zh-book-1},注意收到编辑因为的bibkey即可。

可以将文献标注为右上角\citess{shiweisong2019},只需要在现有的cite后加“ss”即可。

英文会议\cite{Kraus2021Current}, \cite{WuYangLuEtAl2021}.

英文期刊\cite{LuoZengYuanEtAl2016}, \cite{Wu2022Boosting}.


\textbf{特别强调:}从Google下载的bib也不一定全是对的,如发现有信息缺失,请下载原文核对。比如已发表的期刊,要包保证年、卷、标。

\textbf{注意:}如发现替换后的参考文献没有更新,请删除主文件夹下xxx.bbl文件,重新编译即可。

\newpage


\section{总结与展望}

\noindent{纯数字编号}
\begin{enumerate}
 \item XXXXXXXXXX
 \label{item1}
 \item XXXXXXXXXX
 \item XXXXXXXXXX
\end{enumerate}
罗马编号
\begin{enumerate}[label=(\roman*)]
 \item XXXXXXXXXX
 \label{item2}
 \item XXXXXXXXXX
 \item XXXXXXXXXX
\end{enumerate}
括号编号
\begin{enumerate}[label=(\arabic*)]
 \item XXXXXXXXXX
 \label{item3}
 \item XXXXXXXXXX
 \item XXXXXXXXXX
\end{enumerate}
半括号编号
\begin{enumerate}[label=\arabic*)]
 \item XXXXXXXXXX
 \label{item4}
 \item XXXXXXXXXX
 \item XXXXXXXXXX
\end{enumerate}
小字母编号
\begin{enumerate}[label=\alph*)]
 \item XXXXXXXXXX
 \label{item5}
 \item XXXXXXXXXX
 \item XXXXXXXXXX
\end{enumerate}

引用测试,正如\ref{item1}、\ref{item2}、\ref{item3}、\ref{item4}、\ref{item5}所示

\subsection{工作展望}
手动编号 %(不推荐,无法被交叉引用)
\par
本课题针对XX,鉴于XXX,对XX进行了提高,但是XXX,所以有如下XX:

(1)目前XX虽然XX,但是XX仍然XX,所以XX仍然是一个值得XX的问题。

(2)随着XX,XX具有XX的问题,仍值得进一步XX。

(3)本课题在XX有了XX,但是XX的XX还存在XX,所以XX。


\newpage
